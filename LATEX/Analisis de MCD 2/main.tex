\documentclass[11pt, conference]{IEEEtran}
\usepackage[spanish]{babel}
\usepackage[utf8]{inputenc}
\usepackage{amsmath}
\usepackage{amsfonts}
\usepackage{cite}
\usepackage{graphicx}

\begin{document}
	\title{\bf Implementación y Análisis de Código}
	\author{Kevin Jhomar Sanchez Sanchez}

\maketitle

\section{Algoritmo de MCD}
\subsection{Código}
$1.\ ZZ\ euclides(ZZ\  a,\  ZZ\  b)\{\\
2.\qquad ZZ\  r,\  d;\\
3.\qquad r\  =\  rsta\  =\  1;\\
4.\qquad while(r\  >\  0)\{\\
5.\qquad\qquad  d\  =\  r;\\
6.\qquad\qquad  r\  =\  mod(a,b);\\
7.\qquad\qquad  a\  =\  b;\\
8.\qquad\qquad  b\  =\  r;\\
9.\qquad \}\\
10.\qquad return\  d;\\
11.\ \}$
\
\subsection{Análisis}
\begin{center}
	\begin{tabular}{|c|c|c|}
		\hline
		\textbf{Bits}& {\bf Tiempo} (ms) & {\bf Vueltas} \\	\hline
		$10$ & 0.111 & 10\\ \hline
		$50$ & 0.181 & 35 \\ \hline
		$100$ & 0.229 & 63 \\ \hline
		$200$ & 0.480 & 124 \\ \hline
		$400$ & 0.668 & 232 \\ \hline
		$600$ & 1.147 & 297 \\ \hline
		$800$ & 1.321 & 461 \\ \hline
		$1023$ & 1.542 & 465 \\ \hline
	\end{tabular}
\end{center}

\subsection{Seguimiento del algoritmo}

\begin{center}
	\begin{tabular}{|c|c|c|c|c|}
		\hline
		& \textbf{a}& \textbf{b} & \textbf{r} & \textbf{d} \\	\hline
		Inicio & 1827 & 1024 & 1 & 1\\ \hline
		1 & 1024 & 803 & 803 & 803\\ \hline
		2 & 803 & 221 & 221 & 221\\ \hline
		3 & 221 & 140 & 140 & 140\\ \hline
		4 & 140 & 81 & 81 & 91\\ \hline
		\vdots & \vdots & \vdots & \vdots & \vdots\\ \hline
		9 & 15 & 7 & 7 & 7\\ \hline
		10 & 7 & 1 & 1 & 1\\ \hline
	\end{tabular}
\end{center}
Donde:\\
$
a\ :\ Entrada \\
b\ :\ Entrada \\
r\ :\ Residuo \\
d\ :\ Es\ el\ MCD \\
$

\section{Algoritmo de Resto Mínimo}
\subsection{Código}
$1.\ ZZ\ resto\_minimo(ZZ\ a,\ ZZ\ b)\{\\
2.\qquad ZZ\ c,d,r;\\
3.\qquad if(a\ ==\ 0)\{\\
4.\qquad\qquad return\ b;\\
5.\qquad \}else\{\\
6.\qquad\qquad while(b\ !=\ 0)\{\\
7.\qquad\qquad\qquad r\ =\ mod(a,b);\\
8.\qquad\qquad\qquad a\ =\ b;\\
9.\qquad\qquad\qquad b\ =\ r;\\
10.\qquad\qquad\}\\
11.\qquad\}\\
12.\qquad return\ a;\\
13.\ \}$
\
\subsection{Análisis}
\begin{center}
	\begin{tabular}{|c|c|c|}
		\hline
		\textbf{Bits}& {\bf Tiempo} (ms) & {\bf Vueltas} \\	\hline
		$10$ & 0.071 & 10\\ \hline
		$50$ & 0.115 & 35 \\ \hline
		$100$ & 0.114 & 63 \\ \hline
		$200$ & 0.124 & 124 \\ \hline
		$400$ & 0.348 & 232 \\ \hline
		$600$ & 0.300 & 297 \\ \hline
		$800$ & 0.660 & 461 \\ \hline
		$1023$ & 0.671 & 465 \\ \hline
	\end{tabular}
\end{center}
\subsection{Seguimiento del algoritmo}

\begin{center}
	\begin{tabular}{|c|c|c|c|}
		\hline
		& \textbf{a}& \textbf{b} & \textbf{r} \\	\hline
		Inicio & 1827 & 1024 & 803\\ \hline
		1 & 1024 & 803 & 221\\ \hline
		2 & 803 & 221 & 140\\ \hline
		3 & 221 & 140 & 59\\ \hline
		4 & 81 & 59 & 22\\ \hline
		\vdots & \vdots & \vdots & \vdots \\ \hline
		8 & 15 & 7 & 1\\ \hline
		9 & 7 & 1 & 0\\ \hline
	\end{tabular}
\end{center}
Donde:\\
$
a\ :\ Entrada \\
b\ :\ Entrada \\
r\ :\ Residuo \\
$

\section{Algoritmo LSBGCD}
\subsection{Código}
$1.\ ZZ\ LSBGCD(ZZ\ a,\ ZZ b)\{\\
2.\qquad ZZ\ t\ =\ (ZZ)0;int\ s\ =\ 0;\\
3.\qquad while(b\ !=\ (ZZ)0)\{\\
4.\qquad\qquad s\ =\ 0;\\
5.\qquad\qquad while((b<<s)\ <=\ a)\ s++;\\
6.\qquad\qquad --s;\\
7.\\		
8.\qquad\qquad t\ =\ min(a-(b<<s),(b<<s+1)-a);\\
9.\qquad\qquad a\ =\ b;\ b\ =\ t;\\
10.\qquad\qquad if(b\ >\ a)\{\\
11.\qquad\qquad\qquad ZZ\ temp\ =\ a;\\
12.\qquad\qquad\qquad a\ =\ b;\\
13.\qquad\qquad\qquad b\ =\ temp;\\
14.\qquad\qquad\}\\
15.\qquad\}\\
16.\qquad return\ a;\\
17.\ \}$
\
\subsection{Análisis}
\begin{center}
	\begin{tabular}{|c|c|c|}
		\hline
		\textbf{Bits}& {\bf Tiempo} (ms) & {\bf Vueltas} \\	\hline
		$10$ & 0.092 & 9\\ \hline
		$50$ & 0.227 & 39 \\ \hline
		$100$ & 0.348 & 69 \\ \hline
		$200$ & 0.676 & 136 \\ \hline
		$400$ & 1.331 & 270 \\ \hline
		$600$ & 3.149 & 357 \\ \hline
		$800$ & 2.769 & 516 \\ \hline
		$1023$ & 53.765 & 587 \\ \hline
	\end{tabular}
\end{center}
\subsection{Seguimiento del algoritmo}
\begin{center}
	\begin{tabular}{|c|c|c|c|c|}
		\hline
		& \textbf{a}& \textbf{b} & \textbf{s} & \textbf{t} \\	\hline
		Inicio & 1827 & 1024 & 0 & 0\\ \hline
		1 & 1024 & 221 & 0 & 221\\ \hline
		2 & 221 & 140 & 2 & 140\\ \hline
		3 & 140 & 59 & 0 & 59\\ \hline
		4 & 59 & 22 & 1 & 22\\ \hline
		\vdots & \vdots & \vdots & \vdots \\ \hline
		7 & 7 & 1 & 1 & 1\\ \hline
		8 & 1 & 1 & 2 & 1\\ \hline
	\end{tabular}
\end{center}
Donde:\\
$
a\ :\ Entrada \\
b\ :\ Entrada \\
s\ :\ Incrementador \\
t\ :\ Variable Temporal \\
$

\section{Algoritmo de MCD Binario}
\subsection{Código}

\
$1.\ ZZ\ euclides\_binary(ZZ\ a,\ ZZ\ b)\{\\
2.\qquad int\ i;\\
3.\qquad if(a\ ==\ 0)\ return\ b;\\
4.\qquad else\ if(b\ ==\ 0)\ return\ a;\\
5.\qquad else\ if(b\ ==\ a)\ return\ a;\\
6.\qquad else\{\\
7.\qquad\qquad for(i=0;(a\&1)==0\ \&\&\ (b\&1)==0;i++)\{\\
8.\qquad\qquad\qquad a\ =\ a>>1;\\
9.\qquad\qquad\qquad b\ =\ b>>1;\\
10.\qquad\qquad \}\\
11.	\\	
12.\qquad\qquad while((a\ \&\ 1)\ ==\ 0)\ a\ =\ a>>1;\\
13.\qquad\qquad while(b\ !=\ 0)\{\\
14.\qquad\qquad while((b\ \&\ 1)\ ==\ 0)\ b\ =\ b>>1;\\
15.\qquad\qquad\qquad if(a\ >\ b)\{\\
16.\qquad\qquad\qquad\qquad ZZ\ t\ =\ b;\\
17.\qquad\qquad\qquad\qquad b\ =\ a;\\
18.\qquad\qquad\qquad\qquad a\ =\ t;\\
19.\qquad\qquad\qquad \}\\
20.\qquad\qquad\qquad b\ =\ b\ -\ a;\\
21.	\qquad\qquad\}\\
22.\qquad\qquad return\ a\ <<\ i;\\
23.\qquad\}\\
24.\ \}$

\
\subsection{Análisis}
\begin{center}
	\begin{tabular}{|c|c|c|}
		\hline
		\textbf{Bits}& {\bf Tiempo} (ms) & {\bf Vueltas} \\	\hline
		$10$ & 0.083 & 22\\ \hline
		$50$ & 0.107 & 110 \\ \hline
		$100$ & 0.334 & 223 \\ \hline
		$200$ & 0.426 & 417 \\ \hline
		$400$ & 0.363 & 850 \\ \hline
		$600$ & 0.386 & 1242 \\ \hline
		$800$ & 0.698 & 1641 \\ \hline
		$1023$ & 0.977 & 2136 \\ \hline
	\end{tabular}
\end{center}

\

\textbf{NOTA:} Se han aplicado las propiedades binarias.
\begin{itemize}
	\item Cuando 'a' y 'b' son pares. \\
		  $2*MCD(a/2,b/2)$
	\item Cuando 'a' es par y 'b' es impar o viceversa. \\
		  $MCD(a/2,b)$
	\item Cuando 'a' y 'b' son impares o viceversa. \\
		  $MCD(a-b,b)$
\end{itemize}

\pagebreak

\subsection{Seguimiento del algoritmo}

\begin{center}
	\begin{tabular}{|c|c|c|c|}
		\hline
		& \textbf{a}& \textbf{b} & \textbf{t}\\	\hline
		Inicio & 1827 & 1024 & 0\\ \hline
		1 & 1024 & 803 & 1024\\ \hline
		2 & 803 & 221 & 803 \\ \hline
		4 & 221 & 582 & 221 \\ \hline
		5 & 221 & 291 & 221 \\ \hline
		6 & 221 & 70 & 221 \\ \hline
		7 & 221 & 35 & 221 \\ \hline
		\vdots & \vdots & \vdots& \vdots\\ \hline
		21 & 1 & 1 & 1\\ \hline
		22 & 1 & 0 & 1\\ \hline
	\end{tabular}
\end{center}
Donde:\\
$
a\ :\ Entrada \\
b\ :\ Entrada \\	
t\ :\ Variable\ Temporal
$

\section{Algoritmo de Euclides Extendido}
\subsection{Código}
\
$
1.\ vector<ZZ>\ MCD\_extended(ZZ\ a, ZZ\ b)\{\\
2.\qquad ZZ\ r,\ d,\ q,\ x,\ y;\\
3.\qquad x\ =\ y =\ r\ =\ d\ =\ 0;\\
4.\qquad if(a\ !=\ 0\ \&\ b\ ==\ 0)\{\\
5.\qquad\qquad x\ =\ 1;y\ =\ 0;d\ =\ a;\\
6.\qquad \}else\ if(a\ ==\ 0 \& b\ !=\ 0)\{\\
7.\qquad\qquad x\ =\ 0;y\ =\ 1;d\ =\ b;\\
8.\qquad\}else\{\\
9.\qquad\qquad ZZ\ x1\ =\ (ZZ)0,\ x2\ =\ (ZZ)1,\\ 
10.\qquad\qquad\ \ \ \ y1\ =\ (ZZ)1,\ y2\ =\ (ZZ)0;\\
10.\qquad\qquad while(b\ >\ 0)\{\\
11.\qquad\qquad\qquad q\ =\ a/b;\\
12.\qquad\qquad\qquad r\ =\ a\ -\ q*b;\\
13.\qquad\qquad\qquad x\ =\ x2\ -\ q*x1;\\
14.\qquad\qquad\qquad y\ =\ y2\ -\ q*y1;\\
15.\qquad\qquad\qquad a\ =\ b;\\
16.\qquad\qquad\qquad b\ =\ r;\\
17.\qquad\qquad\qquad x2\ =\ x1;\\
18.\qquad\qquad\qquad x1\ =\ x;\\
19.\qquad\qquad\qquad y2\ =\ y1;\\
20.\qquad\qquad\qquad y1\ =\ y;\\
21.\qquad\qquad\}\\
22.\qquad\qquad d\ =\ a;\\
23.\qquad\qquad x\ =\ x2;\\
24.\qquad\qquad y\ =\ y2;\\
25.\qquad \}\\
26.\qquad vector<ZZ> n;\\
26.\qquad n.push\_back(d);n.push\_back(x);n.push_back(y);\\
27.\qquad return\ n;\\
28.\}
$
\subsection{Análisis}
\begin{center}
	\begin{tabular}{|c|c|c|}
		\hline
		\textbf{Bits}& {\bf Tiempo} (ms) & {\bf Vueltas} \\	\hline
		$10$ & 0.152 & 10\\ \hline
		$50$ & 0.278 & 35 \\ \hline
		$100$ & 0.406 & 63 \\ \hline
		$200$ & 1.003 & 124 \\ \hline
		$400$ & 1.260 & 232 \\ \hline
		$600$ & 0.810 & 297 \\ \hline
		$800$ & 1.957 & 461 \\ \hline
		$1023$ & 1.576 & 465 \\ \hline
	\end{tabular}
\end{center}

\subsection{Seguimiento del algoritmo}

\begin{center}
	\begin{tabular}{|c|c|c|c|c|c|c|c|}
		\hline
		   & \textbf{q} & \textbf{x} & \textbf{y} & \textbf{x1} & \textbf{x2}& \textbf{y1}& \textbf{y2} \\	\hline
		Inicio &  0 & 0 & 0 & 0 & 1 & 1 & 0\\ \hline
		1 & 1024 & 1 & -1 & 1 & 0 & -1 & 1\\ \hline
		2 & 803  & -1 & 2 & -1 & 1 & 2 & -1\\ \hline
		3 & 221  & 4 & -7 & 4 & -1 & -7 & 2\\ \hline
		4 & 140  & -5 & 9 & -5 & 4 & 9 & -7\\ \hline
		\vdots  & \vdots & \vdots & \vdots& \vdots& \vdots& \vdots& \vdots \\ \hline
		9 & 15  & -51 & 91 & -51 & 37 & 91 & -66\\ \hline
		10& 7  & 139 & -248 & 139 & -51 & -248 & 91\\ \hline
	\end{tabular}
\end{center}
Donde:\\
$
q\ :\ Cociente \\
x\ :\ Residuo \\
y\ :\ Residuo \\
x1\ :\ Dividendo \qquad x2\ :\ Divisor \\
y1\ :\ Dividendo \qquad y2\ :\ Divisor \\
$

\onecolumn
\section{Algoritmo de MCD Binario Extendido}
\subsection{Código}
$1.\ vector<ZZ>\ binary\_extended(ZZ\ x,\ ZZ\ y)\{\\
2.\qquad ZZ\ res;\\
3.\qquad int\ g\ =\ 0;\\
4.\qquad if(x\ ==\ 0)\ res\ =\ y;\\
5.\qquad else\ if(y\ ==\ 0)\ res\ =\ x;\\
6.\qquad else\ if(y\ ==\ x)\ res\ =\ x;\\
7.\qquad else\{\\
8.\qquad\qquad for(int\ i=0;(x\&1)==0\ \&\&\ (y\&1)==0;i++)\{\\
9.\qquad\qquad\qquad\ x\ =\ x>>1;\\
10.\qquad\qquad\qquad y\ =\ y>>1;\\
11.\qquad\qquad\qquad g++;\\
12.\qquad\qquad \}\\
13.\\		
14.\qquad\qquad ZZ\ u\ =\ x,\ v\ =\ y,\ A\ =\ (ZZ)1,\ B\ =\ (ZZ)0,\ C\ =\ (ZZ)0,\ D\ =\ (ZZ)1;\\
15.\qquad\qquad while(u\ !=\ 0)\{\\
16.\qquad\qquad\qquad while((u\ \& 1)\ ==\ 0)\{\\
17.\qquad\qquad\qquad\qquad u\ =\ u\ >>\ 1;\\
18.\qquad\qquad\qquad\qquad if(mod(A,(ZZ)2)\ ==\ 0\ and\ mod(B,(ZZ)2)\ ==\ 0)\{\\
19.\qquad\qquad\qquad\qquad\qquad A\ =\ A>>1;\\
20.\qquad\qquad\qquad\qquad\qquad B\ =\ B>>1;\\
21.\qquad\qquad\qquad\qquad \}else\{\\
22.\qquad\qquad\qquad\qquad\qquad A\ =\ (A+y)\ >>\ 1;\\
23.\qquad\qquad\qquad\qquad\qquad B\ =\ (B-x)\ >>\ 1;\\
24.\qquad\qquad\qquad\qquad\}\\
25.\qquad\qquad\qquad\}\\
26.\qquad\qquad\qquad while((v\ \&\ 1)\ ==\ 0)\{\\
27.\qquad\qquad\qquad\qquad v\ =\ v\ >>\ 1;\\
28.\qquad\qquad\qquad\qquad if(mod(C,(ZZ)2)\ ==\ 0\ and\ mod(D,(ZZ)2)\ ==\ 0)\{\\
29.\qquad\qquad\qquad\qquad\qquad C\ =\ C>>1;\\
30.\qquad\qquad\qquad\qquad\qquad D\ =\ D>>1;\\
31.\qquad\qquad\qquad\qquad \}else\{\\
32.\qquad\qquad\qquad\qquad\qquad C\ =\ (C+y)\ >>\ 1;\\
33.\qquad\qquad\qquad\qquad\qquad D\ =\ (D-x)\ >>\ 1;\\
34.\qquad\qquad\qquad\qquad \}\\	
35.\qquad\qquad\qquad \}\\
36.	\\
37.\qquad\qquad\qquad	if(u\ >=\ v)\{\\
38.\qquad\qquad\qquad\qquad u\ =\ u\ -\ v;\\
39.\qquad\qquad\qquad\qquad A\ =\ A\ -\ C;\\
40.\qquad\qquad\qquad\qquad B\ =\ B\ -\ D;\\
41.\qquad\qquad\qquad \}else\{\\
42.\qquad\qquad\qquad\qquad v\ =\ v\ -\ u;\\
43.\qquad\qquad\qquad\qquad C\ =\ C\ -\ A;\\
44.\qquad\qquad\qquad\qquad D\ =\ D\ -\ B;\\
45.\qquad\qquad\qquad \}\\
46.\qquad\qquad\}\\
47.\qquad\qquad if(u\ ==\ 0)\{\\
48.\qquad\qquad\qquad x\ =\ C;\\
49.\qquad\qquad\qquad y\ =\ D;\\
50.\qquad\qquad\}\\
51.\\
52.\qquad\qquad res\ =\ v<<g;\\
53.\qquad \}\\
54.\qquad vector<ZZ>\ n;\\
55.\qquad n.push\_back(res);\ n.push\_back(x);\ n.push\_back(y);\\
56.\qquad return\ n;\\
57.\ \}$
\subsection{Análisis}
\begin{center}
	\begin{tabular}{|c|c|c|}
		\hline
		\textbf{Bits}& {\bf Tiempo} (ms) & {\bf Vueltas} \\	\hline
		$10$ & 0.306 & 6\\ \hline
		$50$ & 0.707 & 33 \\ \hline
		$100$ & 0.443 & 78 \\ \hline
		$200$ & 2.070 & 138 \\ \hline
		$400$ & 4.715 & 282 \\ \hline
		$600$ & 4.706 & 431 \\ \hline
		$800$ & 8.185 & 541 \\ \hline
		$1023$ & 10.837 & 724 \\ \hline
	\end{tabular}
\end{center}

\

\textbf{NOTA:} Se han aplicado las propiedades binarias.
\begin{itemize}
	\item Cuando 'a' y 'b' son pares. \\
	$2*MCD(a/2,b/2)$
	\item Cuando 'a' es par y 'b' es impar o viceversa. \\
	$MCD(a/2,b)$
	\item Cuando 'a' y 'b' son impares o viceversa. \\
	$MCD(a-b,b)$
\end{itemize}

\subsection{Seguimiento del algoritmo}
\begin{center}
	\begin{tabular}{|c|c|c|c|c|c|c|c|c|}
		\hline
		& \textbf{x} & \textbf{y} & \textbf{u} & \textbf{v} & \textbf{A}& \textbf{B}& \textbf{C}& \textbf{D} \\	\hline
		Inicio &  1827 & 1024 & 1827 & 1024 & 1 & 0 & 0 & 1\\ \hline
		1 & 1827 & 1024 & 1827 & 1 & -138 & 248 & 139& -248\\ \hline
		2 & 1827  & 1024 & 912 & 1 & -208 & 372 & 139 & -248\\ \hline
		3 & 1827  & 1024 & 56 & 1 & 616 & -1099 & 139 & -248\\ \hline
		4 & 1827  & 1024 & 6 & 1 & 834 & -1488 & 139 & -248\\ \hline
		5 & 1827  & 1024 & 2 & 1 & 278 & -496 & 139 & -248\\ \hline
		6 & 139  & -248 & 0 & 1 & 0 & 0 & 139 & -248\\ \hline
	\end{tabular}
\end{center}
Donde:\\
$
x,\ y\ :\ Son\ los\ valores\ inicial. \\
u,\ v\ :\ Valores\ iniciales\ que\ se\ modifica\ durante\ el\ programa \\
A,B,C,D\ :\ Son\ los\ valores\ fijados\ para\ hallar\ el\ extendido.\\ 
$
\pagebreak
\section{Algoritmo MCD de Lehmer}
\subsection{Código}
$1.\ ZZ\ lehmer\_mcd(ZZ\ A,ZZ\ B,ZZ\ p,ZZ\ system)\{\\ 
2.\qquad ZZ\ a0,a1,b,W,q0,q1,r,u0,u1,v0,v1,R,T;\\
3.\qquad W\ =\ exp(system,p);\\
4.\qquad int\ h;\\
5.\qquad while(B\ >=\ W)\{\\
6.\qquad\qquad conv(h,NumBits(B)-p+1);\\
7.\qquad\qquad a0\ =\ A\ >>\ h;\\
8.\qquad\qquad a1\ =\ B\ >>\ h;\\
9.\qquad\qquad u0\ =\ (ZZ)1;u1\ =\ (ZZ)0;v0\ =\ (ZZ)0;v1\ =\ (ZZ)1;\\
10.\qquad\qquad while((a1+u1\ !=\ (ZZ)0) and (a1+v1\ !=\ (ZZ)0 ))\{\\
11.\qquad\qquad\qquad q0\ =\ (a0+u0)/(a1+u1);\\
12.\qquad\qquad\qquad q1\ =\ (a0+v0)/(a1+v1);\\
13.			\\
14.\qquad\qquad\qquad if(q0\ !=\ q1)\\
15.\qquad\qquad\qquad\qquad break;\\
16.			\\
17.\qquad\qquad\qquad r=a0-q0*a1;\ a0=a1;\ a1=r;\\
18.\qquad\qquad\qquad r=u0-q0*u1;\ u0=u1;\ u1=r;\\
19.\qquad\qquad\qquad r=v0-q0*v1;\ v0=v1;\ v1=r;\\
20.\qquad\qquad \}\\
21.		\\
22.\qquad\qquad if\ (v0\ ==\ (ZZ)0)\{\\
23.\qquad\qquad\qquad R\ =\ mod(A,B);\\
24.\qquad\qquad\qquad A\ =\ B;\\
25.\qquad\qquad\qquad B\ =\ R;\\
26.\qquad\qquad \}else\{\\
27.\qquad\qquad\qquad R\ =\ u0*A+v0*B;\\
28.\qquad\qquad\qquad T\ =\ u1*A+v1*B;\\
29.\qquad\qquad\qquad A\ =\ R;\\
30.\qquad\qquad\qquad B\ =\ T;\\
31.\qquad\qquad \}\\
32.\qquad\qquad if(B\ ==\ 0)\ return\ A;\\
33.\qquad\} \\
34.\qquad A\ =\ euclides(A,B);\\
35.\qquad return\ A;\\
36.\ \}$

\subsection{Análisis}
\begin{center}
	\begin{tabular}{|c|c|c|}
		\hline
		\textbf{Bits}& {\bf Tiempo} (ms) & {\bf Vueltas} \\	\hline
		$10$ & 0.092 & 1\\ \hline
		$50$ & 0.292 & 30 \\ \hline
		$100$ & 0.912 & 58 \\ \hline
		$200$ & 1.201 & 122 \\ \hline
		$400$ & 0.882 & 238 \\ \hline
		$600$ & 1.134 & 315 \\ \hline
		$800$ & 2.321 & 490 \\ \hline
		$1023$ & 4.095 & 487 \\ \hline
	\end{tabular}
\end{center}
\subsection{Seguimiento del algoritmo}

\begin{center}
	\begin{tabular}{|c|c|c|c|c|c|c|c|c|c|c|c|c|c|}
		\hline
		& \textbf{A} & \textbf{B} & \textbf{a0} & \textbf{a1} & \textbf{q0}& \textbf{q1} & \textbf{r} & \textbf{u0} & \textbf{u1} & \textbf{v0} & \textbf{v1} & \textbf{R} & \textbf{T}\\	\hline
		Inicio &  986523541 & 894612341 & 0 & 0 & 0 & 0 & 0& 0& 0& 0& 0& 0& 0\\ \hline
		1 & 894612341 & 91911200 & 3 & 3  &  1 & 0 & 0 & 1 & 0 & 0 & 1 & 91911200 & 0 \\ \hline
		2 & 91911200 & 67411541 & 26 & 2  &  13 & 8 & 0 & 1 & 0 & 0 & 1 & 67411541 & 0 \\ \hline
		3 & 67411541 & 24499659 & 2 & 2  &  1 & 0 & 0 & 1 & 0 & 0 & 1 & 24499659 & 0 \\ \hline
		4 & 24499659 & 18412223 & 8 & 2  &  4 & 2 & 0 & 1 & 0 & 0 & 1 & 18412223 & 0 \\ \hline
		5 & 18412223 & 6087436 & 2 & 2  &  1 & 0 & 0 & 1 & 0 & 0 & 1 & 6087436 & 0 \\ \hline
		6 & 6087436 & 149915 & 8 & 2  &  4 & 2 & 0 & 1 & 0 & 0 & 1 & 149915 & 0 \\ \hline
		7 & 149915 & 90836 & 92 & 2  &  46 & 30 & 0 & 1 & 0 & 0 & 1 & 90836 & 0 \\ \hline
		8 & 90836 & 59079 & 4 & 2  &  2 & 1 & 0 & 1 & 0 & 0 & 1 & 59079 & 0 \\ \hline
		9 & 59079 & 31757 & 5 & 3  &  2 & 1 & 0 & 1 & 0 & 0 & 1 & 31757 & 0 \\ \hline
		10 & 31757 & 27322 & 7 & 3  &  2 & 1 & 0 & 1 & 0 & 0 & 1 & 27322 & 0 \\ \hline
		11 & 27322 & 4435 & 3 & 3  &  1 & 0 & 0 & 1 & 0 & 0 & 1 & 4435 & 0 \\ \hline
		\multicolumn{14}{|c|}{A partir de Aqui se realiza el Algoritmo de Euclides} \\
		\hline		
	\end{tabular}
\end{center}
Donde:\\
$
A,\ B\ :\ Entradas \\
a1,\ a2:\ Son\ las\ entradas\ que\ han\ sido\ desplazadas\ a\ la\ derecha \\
r\ :\ Residuo\\
u0,\ u1,\ v0,\ v1:\  Variables\ Temporales\ del\ residuo\\
R,\ T \ :\ Variables\ Temporales\\
$
\bigskip
\bigskip
\bigskip
\bigskip
\bigskip
\bigskip
\pagebreak
\section{Conclusión}
Bueno como hemos visto cada algoritmo tiene lo suyo ya que algunos mientras más bits tengan las variables de entrada dan menos vueltas y se demoran menos, otros es todo lo contrario mientras más bits tengan las variables de entrada se demoran más. Pienso que cada algoritmo se tiene que usar para un caso en específico del que se esta desarrollando. A continuación presento las  tablas de análisis finales de todos los algoritmos visto.  

\

\textbf{Tabla de Análisis por Tiempo de Respuesta (ms)}

\

\begin{tabular}{|c|c|c|c|c|c|c|c|}
	\hline
	Bits & \textbf{MCD} & \textbf{Resto Mínimo} & \textbf{LSBGCD} & \textbf{Binario} & \textbf{MCD Extendido}& \textbf{Binario Extendido} & \textbf{Lehmer} \\	\hline
	10 & 0.111 & 0.071 & 0.092 & 0.083 & 0.152 & 0.306 & 0.092\\ \hline
	50 & 0.181  & 0.115 & 0.227 & 0.107 & 0.278 & 0.707 & 0.292\\ \hline
	100 & 0.229  & 0.114 & 0.348 & 0.334 & 0.406 & 0.443 & 0.912\\ \hline
	200 & 0.480 & 0.124 & 0.676 & 0.426 & 1.003 & 2.070 & 1.201\\ \hline
	400 & 0.668  & 0.348 & 1.331 & 0.363 & 1.260 & 4.715 & 0.882\\ \hline
	600 & 1.147 & 0.300 & 3.159 & 0.386 & 1.520 & 4.706 & 1.134\\ \hline
	800 & 1.321 & 0.660 & 2.769 & 0.698 & 1.957 & 8.185 & 2.321\\ \hline
	1023 & 1.542 & 0.671 & 53.765 & 0.977 & 1.576 & 10.837 & 4.095\\
	\hline		
\end{tabular}

\

\

\textbf{Tabla de Análisis por Número de Iteraciones}

\

\begin{tabular}{|c|c|c|c|c|c|c|c|}
	\hline
	Bits & \textbf{MCD} & \textbf{Resto Mínimo} & \textbf{LSBGCD} & \textbf{Binario} & \textbf{MCD Extendido}& \textbf{Binario Extendido} & \textbf{Lehmer} \\	\hline
	10 & 10 & 10 & 9 & 22 & 10 & 6 & 1\\ \hline
	50 & 35  & 35 & 39 & 110 & 35 & 33 & 30\\ \hline
	100 & 63  & 63 & 69 & 223 & 63 & 78 & 58\\ \hline
	200 & 124 & 124 & 136 & 417 & 124 & 138 & 122\\ \hline
	400 & 232  & 232 & 270 & 850 & 232 & 282 & 238\\ \hline
	600 & 297 & 297 & 357 & 1242 & 297 & 431 & 215\\ \hline
	800 & 461 & 461 & 516 & 1641 & 461 & 541 & 490\\ \hline
	1023 & 465 & 465 & 587 & 2136 & 465 & 724 & 487\\
	\hline		
\end{tabular}

\end{document}